\let\negmedspace\undefined
\let\negthickspace\undefined
\documentclass[journal]{IEEEtran}
\usepackage[a5paper, margin=10mm, onecolumn]{geometry}
%\usepackage{lmodern} % Ensure lmodern is loaded for pdflatex
\usepackage{tfrupee} % Include tfrupee package

\setlength{\headheight}{1cm} % Set the height of the header box
\setlength{\headsep}{0mm}     % Set the distance between the header box and the top of the text

\usepackage{gvv-book}
\usepackage{gvv}
\usepackage{cite}
\usepackage{amsmath,amssymb,amsfonts,amsthm}
\usepackage{algorithmic}
\usepackage{graphicx}
\usepackage{textcomp}
\usepackage{xcolor}
\usepackage{txfonts}
\usepackage{listings}
\usepackage{enumitem}
\usepackage{mathtools}
\usepackage{gensymb}
\usepackage{comment}
\usepackage[breaklinks=true]{hyperref}
\usepackage{tkz-euclide} 
\usepackage{listings}
% \usepackage{gvv}                                        
\def\inputGnumericTable{}                                 
\usepackage[latin1]{inputenc}                                
\usepackage{color}                                            
\usepackage{array}                                            
\usepackage{longtable}                                       
\usepackage{calc}                                             
\usepackage{multirow}                                         
\usepackage{hhline}                                           
\usepackage{ifthen}                                           
\usepackage{lscape}
\begin{document}

\bibliographystyle{IEEEtran}

\title{
%	\logo{
JEE ADVANCED

\large{EE1030}

Chapter 16: APPLICATIONS OF DERIVATIVES
%	}
}
\author{Homa Harshitha Vuddanti

(EE24BTECH11062)
}	

\maketitle

\bigskip

\renewcommand{\thefigure}{\theenumi}
\renewcommand{\thetable}{\theenumi}

SECTION A\\

E. SUBJECTIVE SKILLS
\begin{enumerate}
   
\item Suppose $p\brak{x}=a_{0}+a_{1}x+a_{2}x^{2}+\dots +a_{n}x^{n}.$ If $\abs{p\brak{x}}\leq\abs{e^{x-1}-1}$ for all $x\geq0$, prove that $\abs{a_{1}+2a_{2
}+\dots+na_{n}}\leq1.$

 \hfill{(2000 - 5 Marks)}\\
 \item Let $-1\leq p\leq1.$ Show that the equation $4x^{3}-3x-p=0$ has a unique root in the interval \sbrak{1/2, 1} and identify it.

 \hfill{(2003 - 2 Marks)}\\
 
 \item Find a point on the curve $x^{2}+2y^{2}=6$ whose distance from the line $x+y=7$, is minimum.

 \hfill{(2003 - 4 Marks)}\\
 
\item Using the relation $2\brak{1-\cos x}<x^{2},x\neq0$ or otherwise, prove that $\sin \brak{\tan x}\geq x, \forall x \in\sbrak{0, \frac{\pi}{4}}$.

 \hfill{(2003 - 4 Marks)}\\

\item If the function $f:\sbrak{0,4}\mapsto R$ is differentiable then show that 
 \begin{enumerate}
 
  \item For $a,b \in \brak{0,4},\brak{f\brak{4}}^{2}-\brak{f\brak{0}}^{2}=8f^{\prime}\brak{a}f\brak{b}$
  \item $\int_{0}^{4} f\brak{t}dt=2\sbrak{\alpha f\brak{\alpha^{2}}+\beta f\brak{\beta^{2}}}\forall0<\alpha, \beta <2$ \\
  
\end{enumerate}

\hfill{(2003 - 4 Marks)}\\

\item If $P\brak{1}=0$ and $\frac{dP\brak{x}}{dx}>P\brak{x}$ for all $x\geq 1$ then prove that $P\brak{x}>0$ for all $x>1.$

\hfill{(2003 - 4 Marks)}\\

\item Using Rolle's theorem, prove that there is at least one root in $\brak{45^{1/100},46}$ of the polynomial 
$P\brak{x}=51x^{101}-2323\brak{x}^{100}-45x+1035.$

\hfill{(2004 - 2 Marks)}\\

\item Prove that for $x\in\sbrak{0,\frac{\pi}{2}}, \sin x+2x\geq \frac{3x\brak{x+1}}{\pi}$. Explain the identity if any used in the proof.

\hfill{(2004 - 4 Marks)}\\

\item If $\abs{f\brak{x_{1}}-f\brak{x_{2}}}<\brak{x_{1}-x_{2}}^{2},$ for all $x_{1},x_{2} \in R$. Find the equation of tangent to the curve $y=f(x)$ at the point \brak{1,2}.

\hfill{(2005 - 2 Marks)}\\

\item If $p\brak{x}$ be a polynomial of degree 3 satisfying $p\brak{-1}=10, p\brak{1}=-6$ and $p\brak{x}$ has maxima at $x=-1$ and $p^{\prime}\brak{x}$ has minima at $x=1$. Find the distance between the local maxima and local minima of the curve.

\hfill{(2005 - 4 Marks)}\\

\item For a twice differentiable function $f\brak{x}, g\brak{x}$ is defined as $g\brak{x}=\brak{f^{\prime}\brak{x}^{2}+f^{\prime\prime}\brak{x}} f\brak{x}$ on $\sbrak{a,e}$. If for $a<b<c<d<e, f\brak{a}=0,f\brak{b}=2,f\brak{c}=-1,f\brak{d}=2,f\brak{e}=0$ then find the minimum number of zeros of $g\brak{x}$.

\hfill{(2006 - 6 Marks)}\\  
\end{enumerate}
F. MATCH THE FOLLOWING \\

\hrule \

{\em Each question contains statements given in two columns, which have to be matched. The statements in Column-I are labelled A,B,C and D, while the statements in Column-II are labelled p,q,r,s and t. Any given statement in Column-I can have correct matching with ONE OR MORE statement\brak{s} in Column-II. The appropriate bubbles corresponding to the answers to these questions have to be darkened as illustrated in the following example. If the correct matches are A-p, s and ; B-q and r;  C-p and q; and D-s then the correct darkening of bubbles will look like the given.}\\
\hrule
\begin{enumerate}
    \item In this questions there are entries in column I and II. Each entry in column I is related to exactly one entry in column II. Write the correct letter from column II against the entry number in column I in your answer book.

\hfill{(1992 - 2 Marks)}\\
Let the functions defined in column I have domain $\brak{\frac{-\pi}{2},\frac{\pi}{2}}$\\
\begin{minipage}[t]{0.32\textwidth}
 \textbf{Column 1}\\
 \begin{enumerate}[label=(\Alph*)]
     \item $x+ \sin x$
     \item $\sec x$
 \end{enumerate}
     \end{minipage}
     \hfill
\begin{minipage}[t]{0.32\textwidth}
    \textbf{Column 2}\\
    \begin{enumerate}[label=(\alph*), start=16]
    \item increasing
    \item decreasing
    \item neither increasing nor decreasing
\end{enumerate}
\end{minipage}
\hrule\


\textbf{(Qs. 2-4):} 
By appropriately matching the information given in the three columns of the following table.
Let $f\brak{x} = x+\ln x-x\ln x, x\in\brak{0,\infty}$\\
Column 1 contains information about zeroes of $f\brak{x}, f^{\prime}\brak{x}$ and $f^{\prime\prime}\brak{x}$.\\
Column 2 contains information about the limiting behaviour of $f\brak{x}, f{\prime}\brak{x}$ and $f^{\prime\prime}\brak{x}$ at infinity.\\
Column 3 contains information about increasing/decreasing nature of $f\brak{x}$ and $f{\prime}\brak{x}.$\\


\noindent
\begin{minipage}[t]{0.32\textwidth}
 \textbf{Column 1}\\
 \begin{enumerate}[label=(\Roman*)] 
     \item $f\brak{x}=0$ for some $x \in \brak{1,e^{2}}$
     \item $f^{\prime}\brak{x}=0$ for some $x \in \brak{1,e}$
     \item $f^{\prime}\brak{x}=0$ for some $x \in \brak{0,1}$
     \item $f^{\prime\prime}\brak{x}=0$ for some $x \in \brak{1,e}$
 \end{enumerate}
\end{minipage}
\hfill
\begin{minipage}[t]{0.28\textwidth}
    \textbf{Column 2}\\
\begin{enumerate}[label=(\roman*)] 
\item $\lim_{x \to \infty} f\brak{x} = 0 $
\item $ \lim_{x \to \infty} f\brak{x} = -\infty $
\item $\lim_{x \to \infty} f^{\prime}\brak{x} = -\infty $
\item $\lim_{x \to \infty} f^{\prime\prime}\brak{x} = 0 $
\end{enumerate}
\end{minipage}
\hfill
\begin{minipage}[t]{0.3\textwidth}
    \textbf{Column 3}\\
    
 (P) $f$ is increasing in \brak{0,1} \\
 (Q) $f$ is increasing in $\brak{e,e^{2}}$ \\
 (R) $f^{\prime}$ is increasing in $\brak{0,1} $ \\
 (S) $f^{\prime}$ is decreasing in $\brak{e,e^{2}} $ \\
 
\end{minipage}

\item Which of the following options is the only correct combination?
\begin{enumerate}
    \item \brak{I}\brak{i}\brak{P}
    \item \brak{II}\brak{ii}\brak{Q}
    \item \brak{III}\brak{iii}\brak{R}
    \item \brak{IV}\brak{iv}\brak{S}
\end{enumerate}
    

\item Which of the following options is the only correct combination?
\begin{enumerate}
  \item \brak{I}\brak{ii}\brak{R}
  \item \brak{II}\brak{iii}\brak{S}
  \item \brak{III}\brak{iv}\brak{P}
  \item \brak{IV}\brak{i}\brak{S}
    
\end{enumerate}
    

\item Which of the following options is the only correct combination?
\begin{enumerate}
    \item \brak{I}\brak{iii}\brak{P} 
    \item \brak{II}\brak{iv}\brak{Q}
    \item \brak{III}\brak{i}\brak{R}
    \item \brak{II}\brak{iii}\brak{P}

\end{enumerate}

\end{enumerate}
\end{document}



